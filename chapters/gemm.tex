% Project Specifications

\chapter{General Matrix Multiplication (GEMM)}

The previous chapter discussed about different ways to view the convolution operation and its cousin,
the tranposed convolution, both conceptually and implementationally. When it comes down to implementation,
we have seen that both operations can be implemented with a single matrix multiplication. In other words,
matrix multiplication is the main computational burden of these operations, meanwhile, real world applications
often involve very large matrices. Therefore, as a low-level operation, the implemention of matrix
multiplication is often heavily optimized. General matrix multiplication (GEMM) is considered as the
\textit{de facto} standard operation contained in the Basic Linear Algebra Subprograms (BLAS)
specification.

GEMM is defined as

$$C \leftarrow \alpha op(A) op(B) + \beta C,$$

where $\alpha, \beta \in \mathbb{R}$, $op(X)$ is either $X$ or $X^\intercal$. In our particualr case for transposed convolution, $\alpha = 1$, $\beta = 1$, $op(X) = X$, so GEMM reduces to

$$C \leftarrow A B + C$$

where C is the bias matrix to be added to the product $A B$.

A subtle detail in the implementation of GEMM is the order of storage of matrix entries. There are two
different orders to store the same matrix: row-major order or column-major order. In row-major order,
entries of rows are stored contiguously in memory, while in column-major order, entries of columns are
consecutive to each other in memory.

A näive implementation of GEMM is shown in \ref{code:naivegemm}. Here, $lda$, $ldb$ and $ldc$ are leading
dimension of matrix $A$, $B$ and $C$, respectively.

\begin{code}
  \begin{minted}{c}
  for (int i = 0; i < m; i++)
  {
      for (int j = 0; j < n; j++)
      {
          float sum = 0;
          for (int l = 0; l < k; l++)
            sum += a[l*lda+i] * b[l*ldb+j];
          c[j*ldc+i] = beta * c[j*ldc+i] + alpha * sum;
      }
  }
  \end{minted}
  \captionof{listing}{Näive Implementation of GEMM}
  \label{code:naivegemm}
\end{code}

\clearpage %force the next chapter to start on a new page. Keep that as the last line of your chapter!
