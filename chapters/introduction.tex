% Introduction

\chapter{Introduction}

Specialized hardware for running deep learning algorithms seems to be a natural step in the evolution of
Artificial Intelligence.  Google, for exmaple, developed its own ASIC named Tensor Processing Unit (TPU)
to accelerate tensor computations. The formidable cost of such endeavors limits ASIC development to the big
players in the industry. For tech startups and hobbists, the Field Programmable Gate Array (FPGA) comes to
rescue by filling the gap between high-cost custimzed ICs and the need to make specialized hardware for certain
applications.

Generative models are a class of machine learning algorithms which, instead of processing real world data,
creates fresh and new data the world has never seen before. In other words, such algorithms enables
the machine to paint new paintings, compose new music, and write new poetries.

Generative Adversarial Networks (GANs) are a class of neural networks in which two different networks are
trained to complete against each other to learn about the probability distribution of a particular dataset.
Introduced in 2014 by Ian Goodfellow \textit{et al}., it soon gained polularity in the machine learning
community, kindled a wave of research on improving the training property and quality of generation of GANs.

The marriage of FPGA and GANs seems to be an interesting topic in its own right. The project explores such
possiblities by implementing a pre-trained generator model of GANs on the FPGA board to generate realistic
pictures. A näive version is presented first, then several optimization possibilities are explored. This
paper serves as a rather detailed documentation of the design and implementation process.

The generator is a deep Convolutional Neural Network (CNN), in which a large part of the computation is
done with an operation called the General Matrix Multiplication (GEMM). Therefore, an efficient implementation
of GEMM is crutial to the acceleration. However, CNNs are normally implemented with floating-point numbers,
which are much less efficient to handle in hardware than fixed-point numbers. If only we could carry out
the computation in fixed-point numbers and then convert the result back to floating-point numbers! Such
techniques do exist and they are referred to as quantization, which is the key to realize high performance
in hardware.

\clearpage %force the next chapter to start on a new page. Keep that as the last line of your chapter!
