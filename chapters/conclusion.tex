% Conclusions

\chapter{Conclusions}

This section concludes the project by briefly discussing possible optimization methods.
As mentioned before, the main goal of the design was simplicity, so many optimizations were not applied.
The main bottleneck in the current design, as in many computing systems, is the I/O bandwidth. During
each clock cycle, only one unit of data is read, which is rather inefficient. Therefore, most of the
optimizations are focused on improving I/O efficiency.

The first obvious optimization is to transfer all weight data from Dual-SPI Flash to the \gls{psram} once
the \gls{fpga} is configured. Weights will be loaded from the \gls{psram} subsequently, which has a parallel
interface and will result in much faster loading of weights.
Burst transfer can be used to read data from \gls{psram} and further reduces I/O latency. This is done by
placing the starting address on the address bus, a fixed amount of data is then read repeatedly in a single
``burst''.

Another optimization mentioned is to widen the data buses connected to the \gls{psram}. Currently the
data address of input buffer is only 8-bit. This can be increased to 32-bit or even more. Multiple data
can be loaded at the same time and calculation can be performed in parallel on these data. The current
design is highly sequential and only utilizes around $15\%$ of the \gls{dsp} slices.

Yet another optimization is to further introduce several small caches using distributed RAM. These caches
are implemented using \glspl{lut} and are faster than \glspl{bram}, i.e., they can be read asynchronously.
Once inputs and weights are loaded into these caches, more parallelism can be achieved,
making use of more \gls{dsp} slices.

Finally, on an \gls{fpga} chip with more \gls{bram} capacity, layer-level parallelism can be exploited, i.e.,
several layers can be calculated at the same time. This involves modifying the current ring structure and
forward data to subsequent layers in a single pass. Each layer will work on its input concurrently, but
some global coordination and scheduling is needed in order to avoid data overrun.

\clearpage %force the next chapter to start on a new page. Keep that as the last line of your chapter!
