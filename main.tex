%----------------------------------------------------------------------------------------
%	Metropolia Thesis LaTeX Template
%----------------------------------------------------------------------------------------
% License:
% This work is licensed under the Creative Commons Attribution 4.0 International License. 
% To view a copy of this license, visit http://creativecommons.org/licenses/by/4.0/.
%
% However, this license apply to this template. As a template, it is supposed to be 
% modified for your own needs (with your thesis content). For this reason, if you use 
% this project as a template and not specifically distribute it as part of a another 
% package/program, we grant the extra permission to freely copy and modify these files as 
% you see fit and even to delete this copyright notice. 
% In short, you are free to publish your thesis under whatever license you wish, even 
% keep the all rights reserved to you.
%
% Authors:
% Panu Leppäniemi, Patrik Luoto and Patrick Ausderau
%
% Credits:
% Panu Leppäniemi: abstract, def, cleaning,...
% Patrik Luoto: title page, abstract in Finnish, abbreviation, math,...
% Patrick Ausderau: initial version, style, table of content, bibliography, figure, 
%                   appendix, table, source code listing...
%
% Please:
% If you find mistakes, improve this template and alike, please contribute by sharing 
% your improvements and/or send us your feedback there: 
% https://github.com/panunu/metropolia-thesis-latex
% And of course, if you improve it, add yourself as an author.
%
% Compiler:
% Use XeLaTeX as a compiler.
 
%----------------------------------------------------------------------------------------
%	THESIS INFO
%----------------------------------------------------------------------------------------

% All general information (main language, title, author (you), degree programme, major 
% option, etc.)
% Edit the file chapters/0info.tex to change these information
% Global information (title of your thesis, your name, degree programme, major, etc.) 

% Yifan: keep it as finnish, otherwise there'll be compilation error
\def\thesislang{finnish} %change this depending on the main language of the thesis.
% "english" is the only other supported language currently. If someone has the swedish, please contribute!

\def\secondlang{english} %if the main language is Finnish (or Swedish), you must have 2 abstracts (one in Finnish (or Swedish) and one in English)
%If the main language is English and that you are native Finnish (or Swedish) speaker, you must have also abstract in your native language on top of the English one.

\author{Yifan Yu} %your first name and last name
\def\thesis{Thesis}%keep the half based on the main thesis language
%was Opinnäytetyö

\def\alaotsikko{} %if you don't have subtitle, empty {} it (but don't delete that line)

%Finnish section, for title/abstract
\def\otsikko{Implementing the Generator of DCGAN on FPGA (Draft)}
\def\tutkinto{Bachelor of Engineering} % change to your needs, e.g. "YAMK", etc.
\def\kohjelma{Electronics}
\def\suuntautumis{}
\def\ohjaajat{
Janne Mäntykoski, Senior Lecturer
}
\def\avainsanat{FPGA, Generative Adversarial Networks, Deep Learning}
\def\pvm{\specialdate\today}

%English section, for title/abstract
\title{Implementing the Generator of DCGAN on FPGA}
\def\metropoliadegree {Bachelor of Engineering} % change to your needs, e.g. "master", etc.
\def\metropoliadegreeprogramme {Electronics}
\def\metropoliaspecialisation {}
\def\metropoliainstructors {
Janne Mäntykoski, Senior Lecturer
}
\def\metropoliakeywords {FPGA, Generative Adversarial Networks, Deep Learning}
\date{\longmonth\today}




%----------------------------------------------------------------------------------------
%	GLOBAL STYLES
%----------------------------------------------------------------------------------------

% If you need extra package, etc. modify the style/style.tex file.
% If you are using Windows OS, you will need to change default font to Arial in that 
% style/style.tex file (or install Liberation Sans font to your system).
% If you are using MacOS or linux, make sure you have Liberation Sans font installed.
% Global style. Normally should not be edited. 
% If you use windows OS, eventually change \setmainfont to Arial
% Check around commit https://github.com/panunu/metropolia-thesis-latex/commit/a0c15ac77bab1a52c59c517a18080938e57bf5ef
% to see how the font files were manually added (after downloading them: https://pagure.io/liberation-fonts/ )

\documentclass[11pt,a4paper,oneside,article]{memoir}
\usepackage[\secondlang,\thesislang]{babel}% finnish english swedish
\usepackage{iflang}
\usepackage{amsmath}
\usepackage{amsfonts}
\usepackage{amssymb}
\usepackage{fontspec}
\usepackage{tocloft}
\usepackage{titlesec}
\usepackage[hyphens]{url}
\usepackage{mathtools}
\usepackage{wallpaper}
\usepackage{datetime}
\usepackage[bookmarksdepth=subsection,hidelinks]{hyperref} % for automagic pdf links for toc, refs, etc.
\usepackage[amssymb]{SIunits}
\usepackage[version=3]{mhchem}
\usepackage{pgfplots} %simple plots etc
\usepackage{pgfplotstable}
\usepackage{tikz} % mindmaps, flowcharts, piecharts, examples at http://www.texample.net/tikz/examples/
\usetikzlibrary{shapes.geometric, arrows}


\renewcommand{\dateseparator}{.}
%condition for adding or not space in TOC
\usepackage{etoolbox}
%for compact list
\usepackage{enumitem}
%for block comment
\usepackage{verbatim}
%for "easier" references
\usepackage{varioref}
%forcing single line spacing in bibliography
\DisemulatePackage{setspace}
\usepackage{setspace}
%including figure (image)
\usepackage{graphicx}
%change the numbering for figure
\usepackage{chngcntr}
%strike trough
\usepackage{ulem}
%euro symbol
\usepackage{eurosym}
%try to count
\usepackage{totcount}
%insert source code
%\usepackage{listings}
%require -8bit -shell-escape in the xelatex compile command
\usepackage[newfloat]{minted}
\setminted{tabsize=2,linenos,breaklines,breaksymbolleft={\quad},baselinestretch=1}
\setmintedinline{breaklines}
\usepackage[justification=justified,singlelinecheck=false]{caption}
\usepackage{color}
%force the width of a table instead of column
\usepackage{tabularx}
\usepackage{booktabs} %why not booktabs? :3
% Abbreviations, acronym and glossary
\usepackage[acronym,nonumberlist,section]{glossaries}%xindy,%toc, ,nomain

\usepackage{float} % For forced figure location with modifier H (\begin{figure}[H])
\usepackage{cite} % Make citations to match Metropolia thesis guide

% change font of links in bibliography to same as other text
\usepackage{url}
\urlstyle{same}

% change punctuation of multiple cites to semicolon instead of comma: [1; 2; 3]
\renewcommand\citepunct{; }

% citep-macro for reference with period inside square brackets [1.]
\newcommand{\citep}[1]{
 \renewcommand\citeright{.]}
 \cite{#1}
 \renewcommand\citeright{]}
}

%set date format to D.M.YYYY
\newdateformat{specialdate}{\THEDAY.\THEMONTH.\THEYEAR}
%set date format to D Month YYYY
\newdateformat{longmonth}{\THEDAY~\monthname[\THEMONTH] \THEYEAR}

\newcommand\tn[1]{\textnormal{#1}} %use \tn instead of \textnormal
\newcommand\reaction[1]{\begin{equation}\ce{#1}\end{equation}} %\reaction{} for chemical reactions

%NORMAL TEXT
%all text, title, etc. in the same font: Arial
%NOTE: fontname is case-sensitive
\setmainfont{Liberation Sans}
%line space
\linespread{1.5}
\AtBeginEnvironment{tabular}{\singlespacing}
%\doublespacing
%margin
\usepackage[top=2.5cm, bottom=3cm, left=4cm, right=2cm, nofoot]{geometry}
\setlength{\parindent}{0pt} %first line of paragraph not indented
\setlength{\parskip}{16.5pt} %one empty line to separate paragraph
%list with small line space separation
\tightlists

%IMAGE - FIGURE
%the figures should be placed in the "illustration" folder
\graphicspath{{illustration/}}
%figure number without chapter (1.1, 1.2, 2.1) to (1, 2, 3)
\counterwithout{figure}{chapter}
%border around images
\setlength\fboxsep{0pt}
\setlength\fboxrule{0.5pt}
%caption font size
\captionnamefont{\small}
\captiontitlefont{\small}
%space after figure caption (and other float elements)
\setlength{\belowcaptionskip}{-7pt}

%TABLE
\counterwithout{table}{chapter}

%SOURCE CODE
\newenvironment{code}{\captionsetup{type=listing}}{}
\IfLanguageName {finnish} {\SetupFloatingEnvironment{listing}{name=Listing}} {}
%\counterwithout{lstlisting}{chapter}
%moved after begin document, otherwise does not compile

%% set this format as the default for lstlisting
%\DeclareCaptionFormat{empty}{}
%\captionsetup[lstlisting]{format=empty}

%TOC
%change toc title
\IfLanguageName {finnish} {\addto{\captionsfinnish}{\renewcommand*{\contentsname}{Contents}}} {}
%remove dots
\renewcommand*{\cftdotsep}{\cftnodots}
%chapter title and page number not in bold
\renewcommand{\cftchapterfont}{}
\renewcommand{\cftchapterpagefont}{}
%sub section in toc
\setcounter{tocdepth}{2}
%subsection numbered
\setcounter{secnumdepth}{2}
\renewcommand{\tocheadstart}{\vspace*{-15pt}}
\renewcommand{\printtoctitle}[1]{\fontsize{13pt}{13pt}\bfseries #1}
\renewcommand{\aftertoctitle}{\vspace*{-22pt}\afterchaptertitle}
%spacing afer a chapter in toc
\preto\section{%
  \ifnum\value{section}=0\addtocontents{toc}{\vskip11pt}\fi
}
%spacing afer a section in toc
\renewcommand{\cftsectionaftersnumb}{\vspace*{-3pt}}
%spacing afer a subsection in toc
\renewcommand{\cftsubsectionaftersnumb}{\vspace*{-1pt}}
%appendix in toc with "Appendix " + num
\IfLanguageName {finnish} {
  \renewcommand*{\cftappendixname}{Appendix\space}
  \renewcommand{\appendixtocname}{Appendices}
}{\renewcommand*{\cftappendixname}{Appendix\space}}
%appendix header
\IfLanguageName {finnish} {\def\appname{Appendix\space}}{\def\appname{Appendix\space}}

%TITLES
%chapter title
%\clearforchapter{\clearpage}
\titleformat{\chapter}
{\fontsize{13pt}{13pt}\bfseries\linespread{1}}%\clearpage
{\thechapter}{.5cm}{}
\titlespacing*{\chapter}{0pt}{.32cm}{9pt}
\titleformat{\section}
{\fontsize{12pt}{12pt}\linespread{1}}
{\thesection}{.5cm}{}
\titlespacing*{\section}{0pt}{14pt}{6pt}
\titleformat{\subsection}
{\fontsize{12pt}{12pt}\linespread{1}}
{\thesubsection}{.5cm}{}
\titlespacing*{\subsection}{0pt}{14pt}{6pt}


%QUOTE
\renewenvironment{quote}
  {\list{}{\rightmargin=0pt\leftmargin=1cm\topsep=-10pt}%
  \item\relax\fontsize{10pt}{10pt}\singlespacing}
  {\endlist}

%BIBLIOGRAPHY
%bibliography title to be "references"
%IF THE TITLE DON'T GET RENAMED PROPERLY, move that line after the \begin{document}
\IfLanguageName {finnish} {\addto{\captionsfinnish}{\renewcommand*{\bibname}{References}}} {\renewcommand\bibname{References}}
\makeatletter %reference list option change
\renewcommand\@biblabel[1]{#1\hspace{1cm}} %from [1] to 1 with 1cm gap
\makeatother %
\setlength{\bibitemsep}{11pt}

%count the appendices (since the chapter counter is reset after \appendix).
%! require to complie 2 times
\regtotcounter{chapter}


\makepagestyle{tiivis}
\makeevenhead{tiivis}{}{}{Tiivistelmä}
\makeoddhead{tiivis}{}{}{Tiivistelmä}

\makepagestyle{abstract}
\makeevenhead{abstract}{}{}{Abstract}
\makeoddhead{abstract}{}{}{Abstract}

% Normally, you do not need to modify the title style. It's content comes from the 
% chapters/0info.tex file.
% TITLE PAGE
% Normally, you should not edit this file.

\makeatletter
\renewcommand{\maketitle}{
\thispagestyle{empty}
\ThisCenterWallPaper{1}{viiva}
%
\vspace*{9.5cm}
\tn{\LARGE\@author\\[22pt]\Huge\IfLanguageName {finnish}{\otsikko}{\@title}\\[22pt]\LARGE\alaotsikko\\[1.75cm]}

\parbox{.7\linewidth}{
\IfLanguageName {finnish}{
  Metropolia Metropolia University of Applied Sciences\\
  \tutkinto \\
  \kohjelma \\
  \thesis\\
  \pvm
} {
  Helsinki Metropolia University of Applied Sciences\\
  \metropoliadegree \\
  \metropoliadegreeprogramme \\
  \thesis\\
  \IfLanguageName {finnish}{\pvm}{\@date} % D.M.YYYY date format for Finnish. D Month YYYY for English
}
}
\ThisLRCornerWallPaper{1}{metropolia}
%
\clearpage
}
\makeatother



%----------------------------------------------------------------------------------------
%	ABBREVIATION AND GLOSSARY
%----------------------------------------------------------------------------------------

% Add/edit all your acronyms, abbreviations, glossary entries, etc. definitions in 
% chapters/0abbr.tex file.
% You can have as many as you wish. Only the ones you use in your text (inserted with 
% \gls{} command) will print in the Glossary/Lyhenteet.
% Generate the glossary
\makeglossaries

% Acronyms, abbreviations, etc. 

\newacronym{html}{HTML}{HyperText Markup Language}
\newacronym{ram}{RAM}{Random Access Memory}


% Glossary entries

\newglossaryentry{latex}
{
    name=\LaTeX{},
    description={Is a mark up language specially suited for scientific documents}
}

\newglossaryentry{maths}
{
    name=mathematics,
    description={Mathematics is what mathematicians do}
}



%----------------------------------------------------------------------------------------
%	DOCUMENT STARTS HERE...
%----------------------------------------------------------------------------------------

\begin{document}
\counterwithout{listing}{chapter}

%----------------------------------------------------------------------------------------
%	TITLE PAGE
%----------------------------------------------------------------------------------------

\input{style/title_headers.tex}
\maketitle
\newpage
%all abstract, table of content and glossary will get the metropolia logo at bottom
\LRCornerWallPaper{1}{footer}

%----------------------------------------------------------------------------------------
%	ABSTRACT / Tiivistelmä
%----------------------------------------------------------------------------------------

% If you are international student writing in English, remove the Finnish abstract.
% If you are Finnish citizen, you must have 2 abstracts, one in Finnish (or Swedish 
% depending on your mother tongue) and one in English regardless of the main language of 
% your thesis.
%\input{chapters/0abstract_fi.tex}
% Abstract in English
%Most probably, you only need to change the text of the abstract. Everything else comes from chapter/0info.tex
%If you do not have any appendix, you may delete \total{chapter} and replace with 0

\pagestyle{abstract}
\begin{otherlanguage}{english}
{\renewcommand{\arraystretch}{2}%
\begin{tabular}{ | p{4,7cm} | p{10,3cm} |}
  \hline
  Author(s) \newline
  Title \newline\newline 
  Number of Pages \newline
  Date
  & 
  \makeatletter
  \@author \newline
  \@title \newline\newline
  \pageref*{LastPage} pages + \total{chapter} appendices \newline %! if no appendices, risk to count total of chapter :D
  \IfLanguageName {finnish} {\foreignlanguage{english}{\longdate\@date}} {\@date}
  \makeatother
  \\ \hline
  Degree & \metropoliadegree
  \\ \hline
  Degree Programme & \metropoliadegreeprogramme
  \\ \hline
  Professional Major & \metropoliaspecialisation
  \\ \hline
  Instructor(s) & \metropoliainstructors
  \\ \hline
  \multicolumn{2}{|p{15cm}|}{\vspace{-22pt}
  The parallel nature of FPGA makes it a promising candidate to accelerate machine learning tasks. The
  purpose of this project was to study the acceleration capabilities of FPGA for deep convolutional
  neural networks.\newline
  
  The project was carried out by implementing a generative model on the Nexys 4 trainer board with an
  Artix-7 FPGA from Xilinx. The pre-trained model is part of the popular Generative Adversarial Networks
  (GANs) which can create realistic images that resemble the training data. The core was written in
  Verilog, but several Xilinx IPs were also used to facilitate the design. Xilinx Vivado 2017.4 was used as
  the development platform. Both fixed-point and floating-point arithmetics were used to achieve a balance
  between efficiency and accuracy.\newline
  
  With simplicity as the main goal of the design, some optimizations were deliberately avoided.
  This paper serves as a detailed documentation of the design and implementation process.
  The core operation of the generative model called transposed
  convolution is described. A method to map network weights and biases from high precision floating-point
  representation to low precision integral representation, known as quantization, is derived. The
  quantization scheme leads to an efficient implementation of the General Matrix Multiplication (GEMM) operation,
  which is at the heart of neural network computations. As a conclusion, possible optimization methods are
  discussed as future work.

  } \\[14cm] \hline
  Keywords & \metropoliakeywords
  \\ \hline
\end{tabular}
}
\end{otherlanguage}
\clearpage



%----------------------------------------------------------------------------------------
%	License? Acknowledgement?
%----------------------------------------------------------------------------------------

% Uncomment next line and edit chapters/0license.tex if you want license in your thesis.
%\input{chapters/0license.tex}

% Uncomment next line and edit chapters/0acknowledgement.tex if you want acknowledgements.
%\input{chapters/0acknowledgement.tex}

%----------------------------------------------------------------------------------------
%	TABLE OF CONTENTS
%----------------------------------------------------------------------------------------

\input{style/toc.tex}

%list of figure, tables would come here if relevant?

%----------------------------------------------------------------------------------------
%	Lyhenteet / Abbreviation
%----------------------------------------------------------------------------------------

% If you don't use abbreviations/glossary, remove the following line.
% Abbreviation and Glossary
% Normally, you don't have to modify this file. Your abbreviations, etc. goes in 
% ../chapters/0abbr.tex file.

\begin{singlespacing}

% \gsladdall would add all terms even if not used in your text.
%\glsaddall

{
	\titleformat{\section}
	{\fontsize{13pt}{13pt}\bfseries\linespread{1}}
	{\thesection}{.5cm}{}
	%Adapt labelwidth (sorry for the ugly hack)
	\setlist[description]{leftmargin=!, labelwidth=4em}
	\IfLanguageName {finnish} {
		\printacronyms[title=List of Abbreviations]
	}{
		\printacronyms[title=Abbreviations]
	}
	\setlist[description]{leftmargin=!, labelwidth=7em}
	\printglossary 
	\setlist[description]{style=standard} % reset settings back to default
}
\end{singlespacing}

\clearpage


%----------------------------------------------------------------------------------------
%	CONTENT
%----------------------------------------------------------------------------------------

\input{style/content.tex}%reset page number to 1, no more logo footer, etc.

% Thesis content if you strictly follow the "Final Year Project guide". Of course, you 
% can adapt to your specific needs (add more chapter, rename them, etc.).
% Introduction

\chapter{Introduction}

Specialized hardware for running deep learning algorithms seems to be a natural step in the evolution of
Artificial Intelligence.  Google, for exmaple, developed its own ASIC named Tensor Processing Unit (TPU)
to accelerate tensor computations. The formidable cost of such endeavors limits ASIC development to the big
players in the industry. For tech startups and hobbists, the Field Programmable Gate Array (FPGA) comes to
rescue by filling the gap between high-cost custimzed ICs and the need to make specialized hardware for certain
applications.

Generative models are a class of machine learning algorithms which, instead of processing real world data,
creates fresh and new data the world has never seen before. In other words, such algorithms enables
the machine to paint new paintings, compose new music, and write new poetries.

Generative Adversarial Networks (GANs) are a class of neural networks in which two different networks are
trained to complete against each other to learn about the probability distribution of a particular dataset.
Introduced in 2014 by Ian Goodfellow \textit{et al}., it soon gained polularity in the machine learning
community, kindled a wave of research on improving the training property and quality of generation of GANs.

The marriage of FPGA and GANs seems to be an interesting topic in its own right. The project explores such
possiblities by implementing a pre-trained generator model of GANs on the FPGA board to generate realistic
pictures. A näive version is presented first, then several optimization possibilities are explored. This
paper serves as a rather detailed documentation of the design and implementation process.

The generator is a deep Convolutional Neural Network (CNN). In such networks a large part of the computation is
done with an operation called the General Matrix Multiplication (GEMM). Therefore, an efficient implementation
of GEMM is crutial to the acceleration. However, CNNs are normally implemented with floating-point numbers,
which are much less efficient to handle in hardware than fixed-point numbers. If only we could carry out
the computation in fixed-point numbers and then convert the result back to floating-point numbers! Such
techniques do exist and they are referred to as quantization, which is the key to realize high performance
in hardware.

\clearpage %force the next chapter to start on a new page. Keep that as the last line of your chapter!

% Project Specifications

\chapter{A Description of GANs}

There are two networks in a GANs model: $D$, the discriminator, and $G$, the generator. $D$ is a regular
classifier which models a funtion $f: X -> Y$ where $X$ are the input examples and $Y = {1, 0}$ are labels
that identify the input examples as "authentic" or not. $G$, on the other hand, learns the probability
distribution that generates the input samples and corresponding labels. If the learning is successful,
$G$ will be able to create new samples from the distribution it has learned.

During the training, we feed two types of examples to $D$: existing training examples and examples generated
by $G$. The task of $D$ is to learn to correctly label both types of examples. Meanwhile, $G$ learns to
generate examples that mimic real world examples. The training process improves the ability of both $D$
and $G$, until eventually the output of $G$ will be indistinguishable from real world examples to $D$. Once
trained, $D$ can be discarded and $G$ can be used in different applications.

In the original paper they are both multilayer perceptrons, however many different network types have been
proposed since then. In this project, $D$ and $G$ are both deep convolutional neural networks which are
suitable for image processing. The training of $D$ and $G$ is done on GPU with floating-point numbers. Since
$D$ is discarded after training, from now on we are only concerned with $G$.

Below is the network structure of $G$:

$G$ consists of five transposed convolutional layers. Except for the last layer, the previous four layers
each are followed by a layer of batch normalization, and a layer of rectified linear units (ReLU) for
activation. The last transposed convolutional layer is followed by a layer of $tanh$ function applied to
each element as activation. The network structure is rather simple compared with many other much larger
networks, such as ResNet which has 152 layers.

\clearpage %force the next chapter to start on a new page. Keep that as the last line of your chapter!

% Project Specifications

\chapter{Transposed Convolutional Layer}

In CNNs, convolutional layer extracts various features from the input, essentially performing a downsampling
operation. Transposed convolutional layer, also known as fractionally strided convolutional layers, or sometime
erroneously as deconvolutional layer, on the other hand performs upsampling on the input. Normally unsampling
is often done with interpolation, but transposed convolution offers a novel approach. Conceptually,
if we run a convolution of stride $f$ backwards, it can be seen as convolution with stride $1/f$, hence the
name fractionally strided convolution.

To understand its operation, let us work though a numerical example. We begin with regular convolution
with an input of $4 \times 4$ matrix $A$:

$$
\begin{matrix}
  1 & 2 & 3 & 4 \\
  4 & 3 & 2 & 1 \\
  1 & 2 & 3 & 4 \\
  4 & 3 & 2 & 1
\end{matrix}
$$

Let the kernel be a $2x2$ matrix $K$:

$$
\begin{matrix}
  1 & 2 \\
  3 & 4
\end{matrix}
$$

Assume the convolution operates with padding of $1$ and stride of $2$, that is, we slide $K$ across the
zero-padded matrix $B$ with a step of $2$. $B$ is shown below:

$$
\begin{matrix}
  0 & 0 & 0 & 0 & 0 & 0 \\
  0 & 1 & 2 & 3 & 4 & 0 \\
  0 & 4 & 3 & 2 & 1 & 0 \\
  0 & 1 & 2 & 3 & 4 & 0 \\
  0 & 4 & 3 & 2 & 1 & 0 \\
  0 & 0 & 0 & 0 & 0 & 0
\end{matrix}
$$

When $K$ is slided across $B$, the overlapping entries in $B$ is called a \textit{patch}. If we view $K$ and
the corresponding patch in $B$ as vectors, at each step, a dot product is computed between them and stored as
an element in the result matrix $C$:

$$
\begin{matrix}
  4 & 8 & 12 \\
  12 & 25 & 13 \\
  8 & 7 & 1
\end{matrix}
$$

In practice, the sliding-window approach is inefficient for implementation. All practical implementations
use a pair of operations called $im2col$ and $col2im$ to wrap a single matrix multiplication. Since
matrix multiplication is such a fundamental operation that its algorithm has been highly optimized over the
decades. To start, flatten $K$ into a row vector $K_{row}$:

$$
\begin{matrix}
  1 & 2 & 3 & 4 \\
\end{matrix}
$$

For each patch in $B$ that $K$ convolves with, the entries in that patch are unrolled into a matrix $B_{col}$
made of column vectors:

$$
\begin{matrix}
  0 & 0 & 0 & 0 & 3 & 1 & 0 & 3 & 1 \\
  0 & 0 & 0 & 4 & 2 & 0 & 4 & 2 & 0 \\
  0 & 2 & 4 & 0 & 2 & 4 & 0 & 0 & 0 \\
  1 & 3 & 0 & 1 & 3 & 0 & 0 & 0 & 0
\end{matrix}
$$

This operation is called $im2col$, namely, image to columns. Now, compute the product $K_{row} * B_{col}$, we
obtain a $1x9$ matrix:

$$
\begin{matrix}
  4 & 18 & 12 & 12 & 25 & 13 & 8 & 7 & 1
\end{matrix}
$$

Finally we "reshape" this matrix to the desired $3 \times 3$ output which is $C$ using the operation $col2im$.
Recall that the precedure described above is how regular convolution would be implemented in practice.
On the other hand, there exists an alternative view of the convolution, also performed with a single
matrix multiplication. This alternative view is impractical for implementation as well, but from which
we can easily reverse the input and output. To see this, first unroll the zero-padded input $B$ and the output
$C$ into vectors:

\setcounter{MaxMatrixCols}{20}

$$
B' =
\begin{matrix}
  0 & \dots & 1 & 2 & 3 & 4 & 0 & \dots & 0 & 4 & 3 & 2 & 1 & \dots & 0
\end{matrix}
$$

$$
C' =
\begin{matrix}
  4 & 18 & 12 & 12 & 25 & 13 & 8 & 7 & 1
\end{matrix}
$$

Then the convolution can be represented as a sparse matrix $M$ of $9 \times 36$ with entries from $K$,
one patch per row:

$$
M =
\begin{matrix}
  1 & 2 & 0 & 0 & 0 & 0 & 3 & 4 & 0 & 0 & 0 & \dots \\
  0 & 0 & 1 & 2 & 0 & 0 & 0 & 0 & 3 & 4 & 0 & \dots \\
  \vdots \\
\end{matrix}
$$

This convolution example maps an input of $4 \times 4$ to an output of $3 \times 3$. 

\clearpage %force the next chapter to start on a new page. Keep that as the last line of your chapter!

% uncomment what you need.
%\input{chapters/projectSpec.tex}
%\input{chapters/methods.tex}
%\input{chapters/theory.tex}
%\input{chapters/solution.tex}
% Conclusions

\chapter{Conclusions and Future Work}

This section concludes the project by briefly discussing possible optimization methods.
As mentioned before, the main goal of the design was simplicity, so many optimizations were not applied.
The main bottleneck in the current design, as in many computing systems, is the I/O bandwidth. During
each clock cycle, only one unit of data is read, which is rather inefficient. Therefore, most of the
optimizations are focused on improving I/O efficiency.

The first obvious optimization is to transfer all weight data from Dual-SPI Flash to the \gls{psram} once
the \gls{fpga} is configured. Weights will be loaded from the \gls{psram} subsequently, which has a parallel
interface and will result in much faster loading of weights.
Burst transfer can be used to read data from \gls{psram} and further reduces I/O latency. This is done by
placing the starting address on the address bus, a fixed amount of data is then read repeatedly in a single
``burst''.

Another optimization mentioned is to widen the data buses connected to the \gls{psram}. Current the
data address of input buffer is only 8-bit. This can be increased to 32-bit or even more. Multiple data
can be loaded at the same time and calculation can be performed in parallel on these data. The current
design is highly sequential and only utilizes around $15\%$ of the \gls{dsp} slices.

Yet another optimization is to further introduce several small caches using distributed RAM. These caches
are implemented using \glspl{lut} and are faster than \glspl{bram}, i.e., they can be read asynchronously.
Once inputs and weights are loaded into these caches, more parallelism can be achieved,
making use of more \gls{dsp} slices.

Finally, on an \gls{fpga} chip with more \gls{bram} capacity, layer-level parallelism can be exploited, i.e.,
several layers can be calculated at the same time. This invovles modifying the current ring structure and
forward data to subsequent layers in a single pass. Each layer will work on its input concurrently, but
some global coordination and scheduling is needed in order to avoid data overrun.

\clearpage %force the next chapter to start on a new page. Keep that as the last line of your chapter!


% Sample content to demonstrate LaTeX command. You will likely delete this line and the 
% next \input{sample/*} lines. You are also safe to delete the sample/ folder and its
% content once you refershed your LaTeX skills. Also check the appendix samples.
\input{sample/1content.tex}
\input{sample/2lorem.tex}
\input{sample/3graph.tex}

%----------------------------------------------------------------------------------------
%	BIBLIOGRAPHY REFERENCES
%----------------------------------------------------------------------------------------

\input{style/biblio.tex}

%----------------------------------------------------------------------------------------
%	APPENDICES 
%----------------------------------------------------------------------------------------

\input{style/appendix.tex}
%force smaller vertical spacing in table of content
%!!! There can be some fun depending if the appendices have (sub)sections or not :D
% You will have to play with these numbers and eventually add the \vspace line  before 
% some \chapter and force another number.
% To add more fun, time to time the table of content get wrong after a build :(
\addtocontents{toc}{\vspace{11pt}}
\pretocmd{\chapter}{\addtocontents{toc}{\protect\vspace{-24pt}}}{}{}

\liite{1}% This is a hack to have right page numbering for each appendix. Make sure to 
	 % use a unique number for each appendix.
% Appendix 
% And demonstrate text references and bibliography references in appendix

\chapter{Appendix: Code Listing of Transposed Convolution Module}\label{appx:verilog_listing}

\begin{code}
  \inputminted{verilog}{code/transposed_convolution.v}
  \captionof{listing}{Implementation of Transposed Convolution}
  \label{code:transposed_convolution}
\end{code}

\clearpage %force the next chapter/appendix to start on a new page. Keep that as the last line of your appendix!
% Sample content to demonstrate appendix in LaTeX. You
% are safe to delete this lines (and the next samples) once you refreshed your LaTeX 
% skills (and safe to delete the sample folder and all its file too).

\addtocontents{toc}{\vspace{11pt}}%fix vertical space for Table of Content
\liite{2}
\input{sample/Xappendix2.tex}

\addtocontents{toc}{\vspace{11pt}}
\liite{3}
\input{sample/X_R_example.tex}


%----------------------------------------------------------------------------------------
%	THIS IS THE END 
%----------------------------------------------------------------------------------------
\end{document}
