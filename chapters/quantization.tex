% Project Specifications

\chapter{Quantization}

%In such networks a large part of the computation is done with an
%operation called the \gls{gemm}. An efficient implementation of \gls{gemm} is crucial to the
%acceleration. However, \glspl{cnn} are normally implemented with floating-point numbers, which are usually less
%efficient to handle in hardware than fixed-point numbers. A class of techniques called quantization can
%be used to carry out the computation in fixed-point numbers and then convert the result back to
%floating-point numbers, realizing high performance in hardware. Chapter 4 provides a detailed discussion
%of the quantization scheme used in this project.

Machine learning programs normally represent numbers as floating-point. Although floating-point numbers
have limited precision and can sometimes lead to numerical instability when the numbers involved get too
small or too large, they are intuitive and easy to work with. But floating-point arithmetic can be rather
expensive to implement in hardware. Modern \glspl{fpga} contain dedicated \gls{dsp} unit, some of which
natively
support floating-point, others require special \gls{ip} cores to implement the relevant
circuitry. \gls{fpga}'s processing power mainly shines on fixed-point. In order to run our model in fixed-point
on \gls{fpga}, it is necessary to convert the model into fixed-point representation, through a process known as
quantization. When the result is computed by \gls{fpga}, it is then converted back to floating-point, which is
naturally called dequantization.

Quantization brings additional benefits with the compression of model size. When 32-bit floating-point
numbers are quantized to 8-bit, the model size is reduced by $75\%$. This cuts down memory bandwidth
requirement and improves memory efficiency. Consequently, the power consumption can be greatly reduced.

One could imagine that quantization would cause a tremendous loss to model precision, for example, when
32-bit floating-point numbers (ranging from $1.175494351 \times 10^{-38}$ to $3.402823466 \times 10^{38}$)
are converted to
8-bit unsigned integers (ranging from 0 to 255). However, one of the peculiar
properties of neural networks is that they are very resilient to noise. Of course this is also one of the
reasons why they are so successful in real world applications. We could view the internal precision loss of
quantization as a form of internal noise, it turned out that the precision loss incurred is rather small.

There exist several different quantization methods. This project adopted the quantization scheme
implemented in Google's \textit{gemmlowp} library, a low precision \gls{gemm} implementation for fixed-point.
The original 32-bit single-precision weights represented in \mintinline{c}{float} are mapped to 
\mintinline{c}{uint8\_t}. \gls{gemm} is then
carried out on the \gls{fpga} board with these 8-bit integers. Intermediate \gls{mac} results are 32-bit
integers to accommodate the accumulation. Eventually the final output of the model is converted back to
$float$.

The quantization works by first identifying the range of the data. In neural networks, the values are usually
distributed within a relatively small range. For instance, for 8-bit quantization, if the minimum of the
weights is $-10.0$ and maximum is $20.0$, then $-10.0$ would be mapped to $0$, and $20.0$ mapped to $255$.
However, there is an additional requirement that can bring great benefits to subsequent computations:
the real value $0$ should be exactly representable. As will be shown, this can be achieved with a small
modification to the mapping equation.

Th quantization scheme can be derived with basic algebra. Define a real number $f$ as the result of an affine
mapping $f = S q + B$, where $q$ is the corresponding quantized integer, $S \in \mathbb{R}$ is a constant
scaling factor, and $B \in \mathbb{R}$ is a constant shift. Next, modify the mapping a bit so that the
real value $0$ is exactly representable:

\begin{equation} \label{eq:dequantization}
  f = S(q - Z),
\end{equation}

where $Z \in \mathbb{N}$ is a constant shift applied to $q$. Since convolution and transposed convolution
both can involve zero-padding, having an exact representation of $0$ avoids the accumulation of errors
(essentially a form of bias). With this form of affine mapping, $0$ is trivially mapped to $q = Z$. $S$ and
$Z$ are called the quantization parameters of this mapping. From equation \ref{eq:dequantization},
it is clear the quantization mapping is done with

\begin{equation} \label{eq:quantization}
  q = \frac{f}{S} + Z
\end{equation}

Let $A \in \mathbb{R}^{m \times k}$, $B \in \mathbb{R}^{k \times n}$, $A B = C \in \mathbb{R}^{m \times n}$,
the corresponding quantization parameters for $A$, $B$ and $C$ are $S_A$ and $Z_A$, $S_B$ and $Z_B$,
$S_C$ and $Z_C$. These parameters are predetermined, since in most cases the range of numbers in these matrices
are known. Let $p$, $q$, $r$ be the entries in the quantized version of $A$, $B$ and $C$. 

For an entry $f_C$ in $C$, $f_C = \sum_{i}^{} A_i B_i$, where $i$ is the index of entries in a particular
row in $A$ and corresponding column in $B$,

\begin{equation}
\begin{split}
  f_C & = \sum_{i} A_i B_i \\
      & = \sum_{i} S_A (p_i - Z_A) S_B (q_i - Z_A) \\
      & = S_A S_B \sum_{i} (p_i - Z_A) (q_i - Z_B)
\end{split}
\end{equation}

Clearly the term $\sum_{i} (p_i - Z_A) (q_i - Z_B)$ is the computational core here, let it be
$\mathfrak{K}$. We have

\begin{equation}
\begin{split}
  r_C & = \frac{f_C}{S_C} + Z_C \\
      & = \frac{S_A S_B}{S_C} \mathfrak{K} + Z_C
\end{split}
\end{equation}

Therefore, once $\mathfrak{K}$ is computed, we can simply multiply it by a constant multiplier
$M = \frac{S_A S_B}{S_C}$, then add the zero offset $Z_C$ to obtain the quantized value $r_C$. The computation
of $\mathfrak{K}$ might seem problematic at first: it contains a subtraction operation for each entry,
which would bring extra overhead to the \gls{gemm} computation. Let us expand this expression:

\begin{equation}
\begin{split}
  \mathfrak{K} & = \sum_{i} (p_i - Z_A) (q_i - Z_B) \\
               & = \sum_{i} (p_i q_i - Z_B p_i - Z_A q_i + Z_A Z_B) \\
               & = \sum_{i} p_i q_i - \sum_{i} Z_B p_i - \sum_{i} Z_A q_i + \sum_{i} Z_A Z_B \\
               & = \sum_{i} p_i q_i - Z_B \sum_{i} p_i - Z_A \sum_{i} q_i + k Z_A Z_B
\end{split}
\end{equation}

Note that $\sum_{i} 1$ is k, the number of columns of $A$ and the number of rows of $B$. We have obtained
four terms here. The first term $\sum_{i} p_i q_i$ is the main term. The second term, $- Z_B \sum{i} p_i$
is the product of $- Z_B$ and the sum of the current row, which only needs to be calculated once and added
to each result of the first term. The third term $- Z_A \sum{i} q_i$ and the fourth term $k Z_A Z_B$ are
similar. The computation can be carried out by computing the first term as the core, and add the rest three
terms one by one.

\section{Handling of Bias}

Biases are added to the 32-bit accumulator result. Recall that $f_C = \sum_{i} A_i B_i = S_A S_B \mathfrak{K}$,
it is clear that biases should be quantized with $S = S_A S_B$ and $Z = 0$. When zero offset is $0$, negative
numbers will remain negative, consequently the type of bias values should be signed 32-bit integers.

\clearpage %force the next chapter to start on a new page. Keep that as the last line of your chapter!
