% Abstract in English
%Most probably, you only need to change the text of the abstract. Everything else comes from chapter/0info.tex
%If you do not have any appendix, you may delete \total{chapter} and replace with 0

\pagestyle{abstract}
\begin{otherlanguage}{english}
{\renewcommand{\arraystretch}{2}%
\begin{tabular}{ | p{4,7cm} | p{10,3cm} |}
  \hline
  Author(s) \newline
  Title \newline\newline 
  Number of Pages \newline
  Date
  & 
  \makeatletter
  \@author \newline
  \@title \newline\newline
  \pageref*{LastPage} pages + \total{chapter} appendices \newline %! if no appendices, risk to count total of chapter :D
  \IfLanguageName {finnish} {\foreignlanguage{english}{\longdate\@date}} {\@date}
  \makeatother
  \\ \hline
  Degree & \metropoliadegree
  \\ \hline
  Degree Programme & \metropoliadegreeprogramme
  \\ \hline
  Professional Major & \metropoliaspecialisation
  \\ \hline
  Instructor(s) & \metropoliainstructors
  \\ \hline
  \multicolumn{2}{|p{15cm}|}{\vspace{-22pt}
  The parallel nature of FPGA makes it a promising candidate to accelerate machine learning tasks. The
  purpose of this project is to demonstrate the acceleration capabilities of FPGA for deep convolutional
  neural networks. The project is carried out by implementing a generative model on the Nexys 4 trainer
  board with an Artix-7 FPGA from Xilinx. The pre-trained model is part of the popular Generative Adversarial
  Networks (GANs) which can create realistic images that resemble the training data. The core is written in
  Verilog, but several IPs are also used to facilitate the design. Xilinx Vivado 2017.4 is used as the
  development platform. \newline
  
  The basic design is able to generate $64 \times 64$ images at $?fps$, an encouraging result with much space
  for optimization. This paper serves as a detailed documentation of the design and implementation process,
  as well as optimization methods. The core operation of the generative model called transposed
  convolution is described. A method to map network weights and biases from high precision floating-point
  representation to low precision integral representation, known as quantization, is discussed. The
  quantization scheme leads to an efficient implementation the General Matrix Multiplication (GEMM) operation,
  which is at the heart of neural network computations. As a conclusion, a brief comparison is made between
  FPGA, CPU, ASIC and GPU for machine learning acceleration.

  } \\[14cm] \hline
  Keywords & \metropoliakeywords
  \\ \hline
\end{tabular}
}
\end{otherlanguage}
\clearpage

