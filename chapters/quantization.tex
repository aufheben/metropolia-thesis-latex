% Project Specifications

\chapter{Quantization}

Machine learning programs normally represent numbers as floating-point. Although floating-point numbers
have limited precision and can sometimes lead to numerical instability when the numbers involved get too
small or too large, they are intuitive and easy to work with. But floating-point arithmetic can be very
expensive to implement in hardware. Modern FPGAs contain dedicated DSP unit, some of which natively
support floating-point, other require special Intellectual Property (IP) cores to implement the relavent
circuitry. FPGA's processing power mainly shines on fixed-point. In order to run our model in fixed-point
on FPGA, it is necessary to convert the model into fixed-point representation, through a process known as
quantization. When the result is computed by FPGA, it is then converted back to floating-point, which is
naturally called dequantization.

One could imagined that quantization would cause a tremendous loss to model precision, for example, when
32-bit floating-point numbers are converted to 8-bit unsigned integers. However one of the peculiar
properties of neural networks is that they are very resilient to noise. Of course this is also one of the
reasons why they so successful in real world applications. We could view the internal precision loss of
quantization as a form of internal noise, it turned out that the precision loss incurred is rather small.

There exist many different quantization methods. In this project, we have adopted the quantization scheme
implemented in Google's \textit{gemmlowp} library, a low precision GEMM implementation for fixed-point. The
original 32-bit single-precision weights represented in $float$ are mapped to $uint8\_t$. GEMM is then
carried out on the FPGA board with these 8-bit integers. Eventually the final output of the model is
converted back to $float$.

To describe the quantization scheme, we begin by defining a real number $f$ as the result of an affine
mapping $f = S q + B$, where $q$ is the corresponding quantized integer, $S \in \mathbb{R}$ is a constant
scaling factor, and $B \in \mathbb{R}$ is a constant shift. Notice the mapping is linear. Now, we need to
modify the mapping a bit so that the real value $0$ is exactly representable:

\begin{equation} \label{eq:dequantization}
  f = S(q - Z),
\end{equation}

where $Z \in \mathbb{N}$ is a constant shift applied to $q$. Recall that convolution and transposed convolution
both can involve zero-padding, having an exact representation of $0$ avoids the accumulation of errors
(essentially a form of bias). With this form of affine mapping, $0$ is trivially mapped to $q = Z$. We call
$S$ and $Z$ the quantization parameters of this mapping. From \ref{eq:dequantization}, it is clear the
quantization mapping is done with

\begin{equation} \label{eq:quantization}
  q = \frac{f}{S} + Z
\end{equation}

Let $A$, $B$ be matrices of real numbers and $A B = C$, the corresponding quantization parameters for $A$,
$B$ and $C$ are $S_A$ and $Z_A$, $S_B$ and $Z_B$, $S_C$ and $Z_C$. Note that these parameters are
predetermined. Let $p$, $q$, $r$ be the entries in the quantized version of $A$, $B$ and $C$. We have
everything at hand to derive the quantization scheme.

For an entry $f_C$ in $C$, $f_C = \sum_{i}^{} A_i B_i$, where $i$ is the index of entries in a particular
row in $A$ and corresponding column in $B$,

\begin{equation}
\begin{split}
  f_C & = \sum_{i}^{} A_i B_i \\
      & = \sum_{i}^{} S_A (p_i - Z_A) S_B (q_i - Z_A) \\
      & = S_A S_B \sum_{i}^{} (p_i - Z_A) (q_i - Z_B)
\end{split}
\end{equation}

Clearly the term $\sum_{i}^{} (p_i - Z_A) (q_i - Z_B)$ is the computational core here, let it be
$\mathfrak{K}$. We have

\begin{equation}
\begin{split}
  r_C & = \frac{f_C}{S_C} + Z_C \\
      & = \frac{S_A S_B}{S_C} \mathfrak{K} + Z_C
\end{split}
\end{equation}

Therefore, once $\mathfrak{K}$ is computed, we can simply multiply it by a constant multiplier
$M = \frac{S_A S_B}{S_C}$, then add the zero offset $Z_C$ to obtain the quantized value $r_C$. The computation
of $\mathfrak{K}$ might seem problematic at first: it contains a subtraction operation for each entry,
which would bring extra overhead to the GEMM computation. Let us expand this expression:

\begin{equation}
\begin{split}
  \mathfrak{K} & = \sum_{i}^{} (p_i - Z_A) (q_i - Z_B) \\
               & = \sum_{i}^{} (p_i q_i - Z_B p_i - Z_A q_i + Z_A Z_B) \\
               & = \sum_{i}^{} p_i q_i - Z_B \sum_{i]^{} p_i - Z_A \sum_{i}^{} q_i + \sum{i} Z_A Z_B
\end{split}
\end{equation}


\clearpage %force the next chapter to start on a new page. Keep that as the last line of your chapter!
