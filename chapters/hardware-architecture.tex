% Project Specifications

\chapter{Hardware Architecture}

Our development platform is Nexys 4 from Digilent with an Artix-7 FPGA chip from Xilinx.

\section{Hardware Resources}

Table \ref{table:hardware_resources} summarizes the hardware resources available on the Nexy 4 board.

% [h] means here, can use [h!] to override internal LaTeX parameters
\begin{table}[h]
  \centering
  \caption{Nexys 4 Resource Specifications}
  \begin{tabular}{lr}
    Logic Slices & 15850 \\
    Block RAM & 4860 Kbits \\
    DSP Slices & 240 \\
    Cellular RAM & 16MB \\
    Quad-SPI Flash & 16MB
  \end{tabular}
  \label{table:hardware_resources}
\end{table}

\section{Calculate Memory Requirement}

Block RAM (BRAM) will be the main active memory for input, output, weight, and bias data. The FPGA chip in use
contains 4860 Kbits of BRAM, which is 607.5KB. The original model size is around 13.7MB, which is
considered very compact when compared with other deep learning models but still very demanding for
our system. Once quantized, the model size is roughly 3.5MB. So it is not possible to fit the entire model
into the BRAM and it is inevitable to interact with external storage devices, e.g., the 16MB Quad-SPI Flash
or the 16MB PSRAM on the Nexys 4 board. It is natural to compute the model in a layer-by-layer fashion: read
the weights and biases of a transposed convolutional layer from the external storage device and compute the
layer, followed by one or more auxiliary layers such as batch normalization or activation layer.

\clearpage %force the next chapter to start on a new page. Keep that as the last line of your chapter!
